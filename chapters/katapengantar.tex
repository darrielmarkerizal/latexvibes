\clearpage
\phantomsection
\addcontentsline{toc}{chapter}{KATA PENGANTAR}
\begin{center}
    \textbf{\large KATA PENGANTAR}\\[3em]
\end{center}
%-----------------------------------------

Puji syukur ke hadirat Allah SWT atas berkat, rahmat, dan karunia-Nya, sehingga penulis dapat menyelesaikan laporan Proyek Akhir dengan judul: \judulid ini. Laporan ini disusun untuk memenuhi salah satu syarat memperoleh gelar Sarjana Terapan pada Program Studi Teknologi Rekayasa Perangkat Lunak, Departemen Teknik Elektro dan Informatika, Sekolah Vokasi, Universitas Gadjah Mada.

Penyusunan Proyek Akhir ini tidak lepas dari bantuan, bimbingan, dan dukungan dari berbagai pihak. Oleh karena itu, pada kesempatan ini penulis ingin menyampaikan ucapan terima kasih yang sebesar-besarnya kepada:

\begin{enumerate}
    \item Allah SWT yang telah memberikan rahmat, hidayah, dan karunia-Nya sehingga penulis dapat menyelesaikan laporan Proyek Akhir ini.
    \item Bapak \koordepartemen, selaku Ketua Departemen Teknik Elektro dan Informatika Sekolah Vokasi Universitas Gadjah Mada.
    \item Bapak \koorprodi, selaku Ketua Program Studi Sarjana Terapan Teknologi Rekayasa Perangkat Lunak.
    \item Bapak \pembimbing, selaku Dosen Pembimbing yang telah meluangkan waktu, tenaga, dan pikiran untuk membimbing dan mengarahkan penulis dalam penyusunan laporan ini.
    \item Kedua orang tua penulis, Bapak Oke Widiyantosa dan Ibu Riris Yulistianti, yang senantiasa memberikan doa tulus, kasih sayang, serta dukungan moral maupun materiil yang tak terhingga.
    \item Adik kandung penulis, Sheerin Desthanya, serta segenap keluarga besar yang selalu memberikan semangat dan keceriaan.
    \item Sahabat dan teman dekat yang selalu ada untuk mendengarkan keluh kesah dan memberikan dukungan semangat kepada penulis.
    \item Teman-teman seperjuangan TRPL angkatan 2022 atas kebersamaan, diskusi, dan kerjasamanya selama masa perkuliahan ini.
    \item Nisa Fredlina Mahardika Saputri, terima kasih karena selalu menjadi pendengar yang baik, serta senantiasa memberikan semangat, motivasi, dan dukungan nyata dalam setiap proses yang penulis lalui.
    \item Abel Makkonen Tesfaye (The Weeknd), yang karya-karya musiknya telah setia menemani penulis agar tetap fokus dan bersemangat selama proses pengerjaan laporan ini.
\end{enumerate}

Penulis menyadari bahwa laporan Proyek Akhir ini masih jauh dari kesempurnaan dan memiliki banyak kekurangan. Oleh karena itu, penulis sangat mengharapkan kritik dan saran yang membangun demi perbaikan di masa mendatang. Akhir kata, semoga laporan Proyek Akhir ini dapat memberikan manfaat bagi pembaca dan pihak-pihak yang membutuhkan.

\begin{flushright}
    Yogyakarta, \tglpengesahan\\[1.25cm]
    \penulis \\
    \nim
\end{flushright}